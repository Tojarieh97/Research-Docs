\documentclass[12pt, a4paper]{article}
\setcounter{secnumdepth}{4}


% A pretty common set of packages
\usepackage{physics}
\usepackage{algorithm2e}
\usepackage{braket}
\usepackage{mathtools}
\usepackage[margin=2.5cm]{geometry}
\usepackage[T1]{fontenc}
\usepackage{graphicx}
\usepackage{amssymb}
\usepackage{amsmath}
\usepackage{color}
\usepackage{booktabs}
\usepackage{multirow}
\usepackage{engord}
\usepackage{soul}
\usepackage{textcomp}
\usepackage{parskip}
\usepackage{setspace}
\usepackage{titlesec}
\usepackage{fancyhdr}
\pagestyle{fancy}
\usepackage[UKenglish]{babel}
\usepackage[UKenglish]{isodate}
\usepackage[skip=2pt,font=footnotesize,justification=centering]{caption}
\usepackage[colorlinks=true,
    linkcolor=black,          % color of internal links
    citecolor=black,        % color of links to bibliography
    filecolor=black,      % color of file links
    urlcolor=black]{hyperref}
\usepackage{amstex}


\renewcommand\bibname{References}
\graphicspath{ {./figures/} }

% Do you prefer Sans Serif fonts?
%\usepackage{sfmath}
%\renewcommand{\familydefault}{\sfdefault} 


% Make some additional useful commands
\newcommand{\ie}{\emph{i.e.}\ }
\newcommand{\eg}{\emph{e.g.}\ }
\newcommand{\etal}{\emph{et al}}
\newcommand{\sub}[1]{$_{\textrm{#1}}$}
\newcommand{\super}[1]{$^{\textrm{#1}}$}
\newcommand{\degC}{$^{\circ}$C}
\newcommand{\wig}{$\sim$}
\newcommand{\ord}[1]{\engordnumber{#1}}
\newcommand{\num}[2]{$#1\,$#2}
\newcommand{\range}[3]{$#1$-$#2\,$#3}
\newcommand{\roughly}[2]{$\sim\!#1\,$#2}
\newcommand{\area}[3]{$#1 \! \times \! #2\,$#3}
\newcommand{\vol}[4]{$#1 \! \times \! #2 \! \times \! #3\,$#4}
\newcommand{\cube}[1]{$#1 \! \times \! #1 \! \times \! #1$}
\newcommand{\figref}[1]{Figure~\ref{#1}}
\newcommand{\eqnref}[1]{Equation~\ref{#1}}
\newcommand{\tableref}[1]{Table~\ref{#1}}
\newcommand{\secref}[1]{Section \ref{#1}}
\newcommand{\XC}{\emph{exchange-correlation}}
\newcommand{\abinit}{\emph{ab initio}}
\newcommand{\Abinit}{\emph{Ab initio}}
\newcommand{\Lonetwo}{L1$_{2}$}
\newcommand{\Dznt}{D0$_{19}$}
\newcommand{\Dtf}{D8$_{5}$}
\newcommand{\Btwo}{B$_{2}$}
\newcommand{\fcc}{\emph{fcc}}
\newcommand{\hcp}{\emph{hcp}}
\newcommand{\bcc}{\emph{bcc}}
\newcommand{\Ang}{{\AA}}
\newcommand{\inverseAng}{{\AA}$^{-1}$}
%\newcommand{\comment}[1]{}
\newcommand{\comment}[1]{\textcolor{red}{[COMMENT: #1]}}
\newcommand{\more}{\textcolor{red}{[MORE]}}
\newcommand{\red}[1]{\textcolor{red}{#1}}


% Change this to modify look of header and footer
\lhead{}
\chead{}
\rhead{}
\lfoot{}
\cfoot{\thepage{}}
\rfoot{}
\renewcommand{\headrulewidth}{0pt}

\begin{document}

    \onehalfspacing

    {\centering
    {\LARGE Analysis of NP and QMA Problems Energy Landscape Using Mutually Unbiased Bases \par}
    {\Medium M.Sc. Research Proposal }

    {\Small Hadar Tojarieh \par }
    {\Small Advisor: Assoc. Prof. Tal Mor \par }
    {\Small Technion - Israel Institute of Technology\\}
    \today
    \par
    }

    \singlespacing

    \begin{abstract}
        In quantum mechanics, Mutually Unbiased Bases (MUB) are sets of quantum states that are maximally distinct
        from one another and exhibit a lack of information when a state from one basis is measured in another basis.
        These MUB represent a powerful tool for characterizing quantum states.
        Our primary objective is to demonstrate the utility of MUB in obtaining a reliable energy landscape
        for optimization problems.
        The energy landscape, as a tool for optimization, potentially plays a crucial role in efficiently identifying
        the global minimum solution while avoiding local minima and barren plateaus when dealing with problems (or cases) that are not NP-hard.
        Our secondary goal, regards the field of not NP-hard problems,
        is integrating the MUB approach with machine learning techniques to enhance the efficiency of the
        optimization process.
        This integration aims to reduce the number of search steps required to reach the optimal
        solution and represents a novel approach to the optimization of NP problems and QMA problems which can be modeled as
        quantum systems.
    \end{abstract}

    \tableofcontents

    \onehalfspacing


    \section{Introduction and Background}\label{sec:introduction-and-background}
    \subsection{Mutually Unbiased Bases}\label{subsec:mutually-unbiased-bases}
Mutually unbiased bases (MUB) are a fundamental concept in quantum information that have been studied extensively in recent years.
They are a set of orthonormal bases in a Hilbert space that are pairwise unbiased, meaning that the inner product between
any two states from different bases is the same \cite{durt2010mutually}.
This property has important implications in quantum mechanics,
including quantum state tomography and quantum error correction.

\subsubsection{Mathematical Definition of MUB}\label{subsubsec:mathematical-definition-of-mub}
The mathematical definition of mutually unbiased bases is given as follows:
Let B1 and B2 be two orthonormal bases in a d-dimensional Hilbert space.
B1 and B2 are said to be mutually unbiased
if and only if the absolute value of the inner product between any two vectors in the two bases is equal to $\frac{1}{\sqrt {d}}$.
This property is important for quantum cryptography,
as it allows for the generation of a secure key by measuring a quantum state in multiple, mutually unbiased bases.
Additionally, the existence of MUB in a d-dimensional Hilbert space has been proven to be linked to the
mathematical structure of the unitary group and the finite Fourier transform over finite fields.

When the dimensionality of the Hilbert space is prime or power of prime, it is possible to construct d+1 MUB.
In addition, for certain composite dimensions, there exist subsets of MUB that satisfy certain conditions,
such as the so-called "symmetric informationally complete positive operator-valued measures" (SIC-POVMs),
which can be used in quantum state tomography and related applications.


\subsubsection{Usages in Quantum Tomography}
Recent research has focused on finding MUB in large-dimensional Hilbert spaces, as well as exploring their potential
uses in quantum state tomography and quantum information science.
Quantum state tomography is the process of learning an unknown quantum state from repeated state preparations and measurements,
and yields optimal knowledge of the state's density matrix. MUB can be used to obtain this information efficiently,
as they provide a way to obtain unbiased estimates of the density matrix elements. In particular, the use of symmetric
informationally complete positive operator-valued measures (SIC-POVMs), which form a special subset of MUB,
has been shown to provide accurate and efficient reconstructions of quantum states in a variety of contexts.
For example, SIC-POVMs have been used in quantum process tomography to reconstruct the dynamics of quantum channels,
as well as in experimental studies of high-dimensional entanglement \cite{ZhuHayashi2020}\cite{GrossEtAl2010}.

MUB can also be used to detect and correct errors in quantum systems.
By measuring the state of the system in multiple MUB,
it is possible to reveal the presence of errors and the specific basis in which the error occurred.
This information can then be used to correct the error and restore the system to its original state.

Additionally, MUB have been proposed as a tool for quantum key distribution (QKD). In QKD,
a key is shared between two parties in a secure way, and MUB can be used to encrypt the key in such a way that any
third party trying to intercept it will not be able to read it.
MUB can be used to encrypt the key in a way that it is more secure against any type of attacks~\cite{wootters1989optimal}.

\subsection{Variational Quantum Algorithms (VQA)}\label{subsec:variational-quantum-algorithms}
Variational quantum algorithms (VQA) are a class of quantum algorithms that utilize a classical optimization loop to
find the optimal parameters of a quantum circuit~\cite{cerezo2021variational}.
VQAs are designed to solve optimization problems and sample problems,
including finding the ground state energy of a quantum system and solving linear systems of equations. One of the key features
of VQAs is their ability to handle large quantum systems with a shallow quantum circuit.

\subsubsection{Variational Quantum Eigensolver (VQE)}
The Variational Quantum Eigensolver (VQE) is a popular and widely used VQA, designed to find the eigenvalues of a quantum Hamiltonian,
which describes the energy of a quantum system. The VQE algorithm consists of a parameterized quantum circuit that is used to generate a
wavefunction that approximates the ground state of the Hamiltonian. The circuit parameters are optimized using a classical optimization
algorithm to minimize the energy expectation value of the wavefunction with respect to the Hamiltonian. This process is repeated until convergence,
at which point the final wavefunction is used to estimate the ground state energy~\cite{peruzzo2014variational}.
VQE has been applied to a wide range of problems in quantum chemistry, including the calculation of molecular energies
and properties, as well as the simulation of materials. In particular, VQE has been shown to be highly effective for
simulating small and medium-sized quantum systems, where exact diagonalization and quantum Monte Carlo methods become infeasible
due to the exponential scaling of the dimension of the quantum state space.

\subsubsection{QAOA}
The Quantum Approximate Optimization Algorithm (QAOA) is a VQA that was introduced in 2014 ~\cite{farhi2014quantum} to address combinatorial optimization problems.
QAOA involves the construction of a parameterized quantum circuit, which is optimized using
classical optimization algorithms to minimize a cost function related to the problem at hand.
The cost function is formulated as a sum of Pauli products, and the QAOA circuit acts as a parameterized
rotation of these products in the computational basis. The QAOA algorithm has been applied to various
optimization problems, including MaxCut, Minimum Vertex Cover, and Max-SAT, among others, demonstrating
its potential for solving combinatorial optimization problems on quantum computers.
The QAOA algorithm has also been shown to be effective for tackling optimization problems in classical
machine learning and operations research, such as portfolio optimization and logistics planning.


\subsection{Energy Landscape}\label{subsec:energy-landscape}
An energy landscape is a concept used in physics and chemistry to describe the potential energy of a system as a function
of its configuration \cite{wales2003energy}. The configuration of a system refers to the positions and orientations of
its components. For example, in a protein, the configuration refers to the positions and orientations of its amino
acids \cite{dill2012protein}. The energy landscape is often represented as a surface or a plot, with energy values on
the vertical axis and the coordinates of the system's configuration on the horizontal axes \cite{wales2003energy}.
The landscape can have different features such as minima, maxima, and saddle points, which correspond to stable,
unstable, and meta-stable states of the system, respectively. A local minimum is a point on the energy landscape where
the energy is lower than that of its immediate surroundings. It corresponds to a stable state of the system,
meaning that small perturbations will not cause the system to move away from that point \cite{wales2003energy}.
A global minimum is a point on the energy landscape that has the lowest energy of all possible configurations,
and it represents the most stable state of the system \cite{wales2003energy}. The energy landscape is a useful concept
for understanding the behavior of systems that can exist in multiple configurations, such as proteins, glasses,
and phase transitions. It helps to visualize the possible configurations of the system and their relative stability.
It is also used to design algorithms for finding the global minimum of the energy function \cite{wales1997global}.

\subsection{Vehicle Routing Problem}\label{subsec:vehicle-routing-problem}

The Vehicle Routing Problem (VRP) is a central problem in the field of transportation and logistics.
It aims to find the most efficient routes for a fleet of vehicles to visit a set of customers,
typically referred to as "stops" or "delivery points", while minimizing the total distance traveled.
VRP is known for being a NP-hard problem \cite{toth2002vehicle}, meaning that it is computationally difficult to find an exact solution
in a reasonable amount of time for large instances of the problem.
Some variants of the VRP are known to be NP-complete, meaning that finding the optimal solution is
computationally intractable as the problem size grows.
One of the most common NP-complete versions of the VRP is the Capacitated Vehicle Routing Problem (CVRP) \cite{Toth2014}.
In the CVRP, a fleet of vehicles with limited capacity must be used to serve a set of customers, each with a known
demand. The objective is to minimize the number of vehicles and the total distance traveled while satisfying
the capacity constraints.
Another common NP-complete version of the VRP is the Vehicle Routing Problem with Time Windows (VRPTW) \cite{Desrochers1992}.
In the VRPTW, each customer has a time window during which they can be serviced, and the objective is to minimize
the total travel time while ensuring that all time windows are satisfied.
Both the CVRP and the VRPTW are NP-complete, meaning that finding the optimal solution for large problem instances is
extremely difficult.

\subsubsection{Classical Computing Approaches for the Solution}
The Vehicle Routing Problem (VRP) has been extensively studied by researchers from various fields, including operations research,
computer science, and mathematics. Researchers have proposed a wide range of solution methods for the VRP, including mathematical programming, heuristics,
and meta-heuristics.
There are several approaches that researchers have used to try and solve the VRP:
\begin{itemize}
    \item Mathematical Programming: Researchers have tried to solve the VRP by formulating it as a mathematical programming problem and using various optimization algorithms to find the optimal solution.
    Due to the complexity of the problem, finding an exact optimal solution for large instances of the VRP is computationally infeasible.
    Therefore, researchers have developed various approximations and optimization algorithms to find near-optimal solutions.
    Some of the most common methods used to solve the VRP include linear programming, integer programming, and constraint programming ~\cite{toth2002overview}.
    e.g.\ constraint programming involves formulating the VRP as a set of constraints and using constraint satisfaction algorithms to find a feasible solution.
    \item Heuristics: Researchers have also developed heuristics and meta-heuristics, such as genetic algorithms, simulated annealing, tabu search, and ant colony optimization, to quickly find near-optimal solutions to the VRP.
    \item Artificial Intelligence: Researchers have also applied various artificial intelligence techniques, such as neural networks and decision trees, to find good solutions to the VRP.
    \item Exact Algorithms: Researchers have also developed exact algorithms, such as branch-and-cut and branch-and-price, to find the optimal solution to the VRP. These algorithms are typically very slow but guarantee an optimal solution~\cite{laporte1987exact}.
    \item Hybrid Approaches: Researchers have also combined multiple approaches, such as combining heuristics and exact algorithms, to find better solutions to the VRP~\cite{hifi2014hybrid}.
\end{itemize}
All of these approaches have been successful to some extent in solving the VRP, but the VRP remains a challenging problem and there is still ongoing research in this area.
The VRP is a complex problem that has many variations and applications, and researchers continue to explore new approaches to solve it.

\subsubsection{Quantum Computing Approaches for the Solution}
One of the quantum computing approaches for solving the VRP is the quantum annealing algorithm.
This algorithm uses a quantum system to find the global minimum of an objective function by gradually reducing the
energy of the system. In the case of the VRP, the objective function represents the total distance or cost of all
the routes, and the quantum system can be used to find the optimal solution in a much faster time compared to classical
optimization algorithms.\\
Another approach is the quantum-inspired optimization algorithms, which mimic the behavior of quantum systems to solve
classical optimization problems. These algorithms have been shown to be effective in finding near-optimal solutions to
the VRP in a much faster time compared to classical optimization algorithms.\\
A recent article applied QAOA to the VRP~\cite{9605345}. In this work, the authors propose a quantum-inspired heuristic
based on QAOA for the VRP and
compare its performance with classical heuristics. The results show that the proposed quantum-inspired heuristic
outperforms the classical heuristics in terms of solution quality and computational time.
There were also VQE solution approaches for VRP~\cite{mohanty2022analysis}.
Analysis of the effects of noise in a gate-based
simulation of an algorithm to solve the vehicle routing problem.
It was found that amplitude damping
noise causes the least impact on the results of an optimized
variational circuit. In contrast bit-phase-flip, depolarising and
phase-flip noise channels had the maximum negative impact.





    \section{Related Work}\label{sec:related-work}
    There are few parers about using quantum annealing to solve those kind of optimization problems, that are intractable
    for classical computers.
    Quantum Annealing (QA) is a method for solving optimization problems that is inspired by the process of annealing in
    metallurgy.
    Like classical simulated annealing, it is a method that uses a random search to find the global minimum
    of an energy function, but it uses the principles of quantum mechanics to guide the search.
    In QA, the optimization problem is mapped to a physical system, called the quantum annealer,
    which consists of a large number of interacting quantum bits (qubits).
    The energy function of the optimization problem is encoded into the Hamiltonian of the quantum annealer,
    and the ground state of the Hamiltonian corresponds to the optimal solution of the problem.

    The quantum annealer is initialized in a high-energy state,
    and then it is slowly cooled down to its ground state by applying a time-dependent Hamiltonian.
    The cooling process is guided by a schedule that controls the rate of change of the Hamiltonian.
    The schedule is designed to avoid getting stuck in local minima and to increase the chances of finding the global minimum.


    \section{Objectives of the Research}\label{sec:objectives-of-the-research}
    The purpose of the M.Sc. research is to configure a way to draw the energy landscape of a quantum system using MUB.
    Through that, a great number of realizations can be reached, for example avoid local minima and find the absolute minimum, avoid barren plateaus\@.
    In our research, we are going to focus on a problem which preoccupies the entire optimization world, the Vehicle Routing Problem.
    Although our research will focus on this problem as our quantum system, one can refer to other NP-hard problem, after the usage of MUB as
    a potential tool for drawing the energy landscape will be demonstrated.
    In the pursuit of our research objective, the utilization of quantum algorithmss and the implementation of a hybrid
    quantum-classical approach may provide significant benefits.
    In particular, the modeling of our problem into
    VQE or QAOA has the potential to offer
    improved approximations and the discovery of larger subclasses of NP-hard problems that are not NP-hard themselves.\\
    Despite the limitations imposed by shallow quantum circuits, such a hybrid solution may still outperform traditional
    classical methods, providing a promising avenue for further exploration and advancement in the field.
    VRP can be modeled as a search for the global minimum of an energy function, when the energy function represents
    the cost of a particular solution, such as the total distance traveled by vehicles or the total time taken to visit
    all customers.
    The configuration space is defined by the possible routes and the assignments of customers to vehicles.
    The global minimum of the energy function corresponds to the optimal solution of the problem\@.\\
    The search in the energy space can be done by using machine learning model or neural networks, for the correct differential
    fluctuations needed in the angles of each quantum eigenstate of the mutually unbiased bases in use.
    The use of machine learning in the context of MUB is motivated by two main considerations.
    Firstly, the exponential growth of the number of MUB states for a large number of qubits necessitates the development
    of techniques for efficiently selecting a portion of these states or selecting them only for a subset of qubits.
    Secondly, even for small numbers of qubits, there is a need for innovative methods to explore the space beyond the
    vicinity of MUB states.
    This approach can significantly reduce the computational resources required for the VQE algorithm,
    making it more scalable for larger problems.
    We think about using MUB to construct energy landscapes of quantum systems using machine learning techniques.
    This can provide new insights into the behavior of the system that can be hard to gain by using other methods.
    The application of machine learning algorithms can help to overcome challenges by allowing for the
    characterization of smaller-scale problems and facilitating the transfer of insights to more complex systems with a
    larger number of qubits.
    In order to gain a comprehensive understanding of the landscape, it is necessary to not only examine the exact
    MUB-states points but also their surrounding regions.
    This can be achieved through two methods:
    By initiating the quantum circuit at each MUB state while incorporating a parametrized circuit that combines the MUB state
    with a standard ansatz.
    By starting at each MUB state and constructing a small-angle ansatz through the implementation of small-angle
    rotations of each qubit and small-angle rotations of the controlled-phase operation.


    \section{Other Research Directions}\label{sec:other-research-directions}
    In addition to our main research topic, Prof. Mor and I are collaborating on other research problems.
    We plan to expand Qandies \cite{lin2020quantum} \cite{lin2021quantum} \cite{mor2022digital} terminology
    usage for explaining other quantum key distribution protocols and quantum phenomenas such as
    entanglement-swapping and delay-choice entanglement-swapping~\cite{ma2016delayed}.

    \bibliography{bibi}
    \bibliographystyle{unsrt}
\end{document}